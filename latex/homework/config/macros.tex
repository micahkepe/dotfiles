% Description: Custom macros for LaTeX documents

%%%%%%%%%%%%%%%%%%%%%%%%%%%%%%%%%%%%%%%%%%%%%%%%%%%%%%%%
% Custom macros for quantum computing
%%%%%%%%%%%%%%%%%%%%%%%%%%%%%%%%%%%%%%%%%%%%%%%%%%%%%%%%

\newcommand{\bloch}[2]{%
    \begin{center}
    \tdplotsetmaincoords{60}{120}
    \begin{tikzpicture}[tdplot_main_coords, scale=2]

        % Sphere with 3D effect and an outer border
        \shade[ball color=white!90,opacity=0.2] (0,0,0) circle (1);

        % Equatorial circle
        \draw[thick] (0,0,0) circle (1);

        % Draw major latitude circles for 3D effect
        \foreach \angle in {0,30,...,150} {
            \tdplotsetrotatedcoords{0}{\angle}{0}
            \draw[tdplot_rotated_coords,black!30] (1,0,0) arc (0:360:1);
        }

        % Main axes with arrows
        \draw[-{Stealth[length=3mm]}, thick, blue] (-1.5,0,0) -- (1.5,0,0) node[below] {\large $x$};
        \draw[-{Stealth[length=3mm]}, thick, blue] (0,-1.5,0) -- (0,1.5,0) node[right] {\large $y$};
        \draw[-{Stealth[length=3mm]}, thick, blue] (0,0,-1.5) -- (0,0,1.5) node[above] {\large $z$};

        % State vector using parameters
        % #1 is theta, #2 is phi (in degrees)
        \draw[red,-{Stealth[length=3mm]}, very thick] (0,0,0) --
            ({sin(#1)*cos(#2)},{sin(#1)*sin(#2)},{cos(#1)});
        \node[red] at
            ({sin(#1)*cos(#2)+0.3},{sin(#1)*sin(#2)+0.2},{cos(#1) + 0.05})
            {\large $|\psi\rangle$};

        % Labels for poles with better visibility
        \fill[black] (0,0,1) circle (0.5pt);
        \node[above=3pt] at (-0.4,0,1.1) {\large $|0\rangle$};

        \fill[black] (0,0,-1) circle (0.5pt);
        \node[below=3pt] at (-0.4,0,-1.1) {\large $|1\rangle$};

        % Additional circular guides to emphasize perspective
        \draw[dashed] (1,0,0) arc (0:180:1 and 0.4);
        \draw (1,0,0) arc (0:-180:1 and 0.4);
    \end{tikzpicture}%
    \end{center}
}

% Math things
\newcommand{\norm}[1]{\left\lVert#1\right\rVert}
\newcommand{\abs}[1]{\left\lvert#1\right\rvert}
\newcommand{\inner}[2]{\left\langle#1,#2\right\rangle}

% Quantum mechanics things
\newcommand{\zero}{\left|0\right\rangle}
\newcommand{\one}{\left|1\right\rangle}

%%%%%%%%%%%%%%%%%%%%%%%%%%%%%%%%%%%%%%%%%%%%%%%%%%%%%%%%
% Custom macros I already had
%
% Adapted from:
%   https://github.com/SeniorMars/dotfiles/blob/main/latex_template/macros.tex
%%%%%%%%%%%%%%%%%%%%%%%%%%%%%%%%%%%%%%%%%%%%%%%%%%%%%%%%

%From M275 ``Topology'' at SJSU
\newcommand{\id}{\mathrm{id}}
\newcommand{\taking}[1]{\xrightarrow{#1}}
\newcommand{\inv}{^{-1}}

%From M170 ``Introduction to Graph Theory'' at SJSU
\DeclareMathOperator{\diam}{diam}
\DeclareMathOperator{\ord}{ord}
\newcommand{\defeq}{\overset{\mathrm{def}}{=}}

%From the USAMO .tex files
\newcommand{\ts}{\textsuperscript}
\newcommand{\dg}{^\circ}
\newcommand{\ii}{\item}

% % From Math 55 and Math 145 at Harvard
\newenvironment{subproof}[1][Proof]{%
\begin{proof}[#1] \renewcommand{\qedsymbol}{$\blacksquare$}}%
{\end{proof}}

\newcommand{\liff}{\leftrightarrow}
\newcommand{\lthen}{\rightarrow}
\newcommand{\opname}{\operatorname}
\newcommand{\surjto}{\twoheadrightarrow}
\newcommand{\injto}{\hookrightarrow}
\newcommand{\On}{\mathrm{On}} % ordinals
\DeclareMathOperator{\img}{im} % Image
\DeclareMathOperator{\Img}{Im} % Image
\DeclareMathOperator{\coker}{coker} % Cokernel
\DeclareMathOperator{\Coker}{Coker} % Cokernel
\DeclareMathOperator{\Ker}{Ker} % Kernel
\DeclareMathOperator{\rank}{rank}
\DeclareMathOperator{\Spec}{Spec} % spectrum
\DeclareMathOperator{\Tr}{Tr} % trace
\DeclareMathOperator{\pr}{pr} % projection
\DeclareMathOperator{\ext}{ext} % extension
\DeclareMathOperator{\pred}{pred} % predecessor
\DeclareMathOperator{\dom}{dom} % domain
\DeclareMathOperator{\ran}{ran} % range
\DeclareMathOperator{\Hom}{Hom} % homomorphism
\DeclareMathOperator{\Mor}{Mor} % morphisms
\DeclareMathOperator{\End}{End} % endomorphism

\newcommand{\eps}{\epsilon}
\newcommand{\veps}{\varepsilon}
\newcommand{\ol}{\overline}
\newcommand{\ul}{\underline}
\newcommand{\wt}{\widetilde}
\newcommand{\wh}{\widehat}
\newcommand{\vocab}[1]{\textbf{\color{blue} #1}}
\providecommand{\half}{\frac{1}{2}}
\newcommand{\dang}{\measuredangle} %% Directed angle
\newcommand{\ray}[1]{\overrightarrow{#1}}
\newcommand{\seg}[1]{\overline{#1}}
\newcommand{\arc}[1]{\wideparen{#1}}
\DeclareMathOperator{\cis}{cis}
\DeclareMathOperator*{\lcm}{lcm}
\DeclareMathOperator*{\argmin}{arg min}
\DeclareMathOperator*{\argmax}{arg max}
\newcommand{\cycsum}{\sum_{\mathrm{cyc}}}
\newcommand{\symsum}{\sum_{\mathrm{sym}}}
\newcommand{\cycprod}{\prod_{\mathrm{cyc}}}
\newcommand{\symprod}{\prod_{\mathrm{sym}}}
\newcommand{\Qed}{\begin{flushright}\qed\end{flushright}}
\newcommand{\parinn}{\setlength{\parindent}{1cm}}
\newcommand{\parinf}{\setlength{\parindent}{0cm}}
\newcommand{\inorm}{\norm_{\infty}}
\newcommand{\opensets}{\{V_{\alpha}\}_{\alpha\in I}}
\newcommand{\oset}{V_{\alpha}}
\newcommand{\opset}[1]{V_{\alpha_{#1}}}
\newcommand{\lub}{\text{lub}}
\newcommand{\del}[2]{\frac{\partial #1}{\partial #2}}
\newcommand{\Del}[3]{\frac{\partial^{#1} #2}{\partial^{#1} #3}}
\newcommand{\deld}[2]{\dfrac{\partial #1}{\partial #2}}
\newcommand{\Deld}[3]{\dfrac{\partial^{#1} #2}{\partial^{#1} #3}}
\newcommand{\lm}{\lambda}
\newcommand{\uin}{\mathbin{\rotatebox[origin=c]{90}{$\in$}}}
\newcommand{\usubset}{\mathbin{\rotatebox[origin=c]{90}{$\subset$}}}
\newcommand{\lt}{\left}
\newcommand{\rt}{\right}
\newcommand{\bs}[1]{\boldsymbol{#1}}
\newcommand{\exs}{\exists}
\newcommand{\st}{\strut}
\newcommand{\dps}[1]{\displaystyle{#1}}

\newcommand{\sol}{\setlength{\parindent}{0cm}\textbf{\textit{Solution:}}\setlength{\parindent}{1cm} }
\newcommand{\solve}[1]{\setlength{\parindent}{0cm}\textbf{\textit{Solution: }}\setlength{\parindent}{1cm}#1 \Qed}
